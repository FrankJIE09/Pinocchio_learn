\documentclass[10pt,a4paper]{article}
\usepackage[UTF8]{ctex}
\usepackage{geometry}
\usepackage{booktabs}
\usepackage{longtable}
\usepackage{array}
\usepackage{listings}
\usepackage{xcolor}
\usepackage{hyperref}
\usepackage{multicol}

\geometry{margin=1.5cm}

% 代码样式设置
\lstset{
    language=Python,
    basicstyle=\ttfamily\small,
    keywordstyle=\color{blue}\ttfamily,
    stringstyle=\color{red}\ttfamily,
    commentstyle=\color{green}\ttfamily,
    breaklines=true,
    frame=single,
    numbers=left,
    numberstyle=\tiny\color{gray},
    tabsize=2
}

\title{Pinocchio vs MuJoCo API 对比}
\author{刚体动力学库 API 参考}
\date{}

\begin{document}

\maketitle

\section{概述}

本文档对比 Pinocchio 和 MuJoCo 两个刚体动力学库的 API 设计和使用方式。两者都采用 Model-Data 分离的设计模式,但在功能定位和 API 细节上有所不同。

\subsection{核心定位}

\begin{itemize}
    \item \textbf{Pinocchio}: 刚体动力学算法库,专注于算法实现和优化计算
    \item \textbf{MuJoCo}: 物理仿真引擎,提供完整的仿真环境和可视化
\end{itemize}

\section{基础 API 对比}

\subsection{模型和数据初始化}

\begin{longtable}{p{5cm}p{6cm}p{6cm}}
\toprule
\textbf{功能} & \textbf{Pinocchio} & \textbf{MuJoCo} \\
\midrule
\endfirsthead
\toprule
\textbf{功能} & \textbf{Pinocchio} & \textbf{MuJoCo} \\
\midrule
\endhead
\bottomrule
\endfoot
\bottomrule
\endlastfoot
导入库 & \texttt{import pinocchio as pin} & \texttt{import mujoco} \\
创建模型 & \texttt{model = pin.Model()} & \texttt{model = mujoco.MjModel.from\_xml\_string(xml)} \\
从文件加载 & \texttt{model = pin.buildModelFromUrdf(filename)} & \texttt{model = mujoco.MjModel.from\_xml\_path(path)} \\
创建数据 & \texttt{data = pin.Data(model)} & \texttt{data = mujoco.MjData(model)} \\
 & \texttt{data = model.createData()} & \\
\bottomrule
\end{longtable}

\subsection{配置和状态}

\begin{longtable}{p{5cm}p{6cm}p{6cm}}
\toprule
\textbf{功能} & \textbf{Pinocchio} & \textbf{MuJoCo} \\
\midrule
\endfirsthead
\toprule
\textbf{功能} & \textbf{Pinocchio} & \textbf{MuJoCo} \\
\midrule
\endhead
\bottomrule
\endfoot
\bottomrule
\endlastfoot
位置变量数 & \texttt{model.nq} & \texttt{model.nq} \\
速度变量数 & \texttt{model.nv} & \texttt{model.nv} \\
配置向量 & \texttt{q} (nq 维) & \texttt{data.qpos} (nq 维) \\
速度向量 & \texttt{v} (nv 维) & \texttt{data.qvel} (nv 维) \\
加速度向量 & \texttt{a} (nv 维) & \texttt{data.qacc} (nv 维) \\
随机配置 & \texttt{pin.randomConfiguration(model)} & \texttt{mujoco.mj\_setState(model, data, qpos, qvel)} \\
中性配置 & \texttt{pin.neutral(model)} & \texttt{data.qpos[:] = model.qpos0} \\
\bottomrule
\end{longtable}

\section{运动学 API}

\subsection{正向运动学}

\begin{longtable}{p{5cm}p{6cm}p{6cm}}
\toprule
\textbf{功能} & \textbf{Pinocchio} & \textbf{MuJoCo} \\
\midrule
\endfirsthead
\toprule
\textbf{功能} & \textbf{Pinocchio} & \textbf{MuJoCo} \\
\midrule
\endhead
\bottomrule
\endfoot
\bottomrule
\endlastfoot
正向运动学 & \texttt{pin.forwardKinematics(model, data, q, [v, [a]])} & \texttt{mujoco.mj\_forward(model, data)} \\
关节位置 & \texttt{data.oMi[joint\_id]} & \texttt{data.xpos[joint\_id]} \\
关节旋转 & \texttt{data.oMi[joint\_id].rotation} & \texttt{data.xmat[joint\_id]} \\
框架位置 & \texttt{data.oMf[frame\_id]} & \texttt{data.xpos[body\_id]} \\
更新框架位置 & \texttt{pin.updateFramePlacements(model, data)} & \texttt{mujoco.mj\_forward(model, data)} \\
\bottomrule
\end{longtable}

\subsection{雅可比矩阵}

\begin{longtable}{p{5cm}p{6cm}p{6cm}}
\toprule
\textbf{功能} & \textbf{Pinocchio} & \textbf{MuJoCo} \\
\midrule
\endfirsthead
\toprule
\textbf{功能} & \textbf{Pinocchio} & \textbf{MuJoCo} \\
\midrule
\endhead
\bottomrule
\endfoot
\bottomrule
\endlastfoot
完整雅可比 & \texttt{pin.computeJointJacobians(model, data, q)} & \texttt{mujoco.mj\_jacBody(model, data, jacp, jacr, body\_id)} \\
 & \texttt{→ data.J} & \\
关节雅可比 & \texttt{pin.getJointJacobian(model, data, joint\_id, ref\_frame)} & \texttt{mujoco.mj\_jacBodyCom(model, data, jacp, jacr, body\_id)} \\
框架雅可比 & \texttt{pin.computeFrameJacobian(model, data, q, frame\_id, ref\_frame)} & \texttt{mujoco.mj\_jacGeom(model, data, jacp, jacr, geom\_id)} \\
雅可比时间导数 & \texttt{pin.computeJointJacobiansTimeVariation(model, data, q, v)} & \texttt{mujoco.mj\_jacBodyCom(model, data, jacp, jacr, body\_id)} \\
 & & \texttt{+ 手动计算时间导数} \\
\bottomrule
\end{longtable}

\section{动力学 API}

\subsection{正向动力学}

\begin{longtable}{p{5cm}p{6cm}p{6cm}}
\toprule
\textbf{功能} & \textbf{Pinocchio} & \textbf{MuJoCo} \\
\midrule
\endfirsthead
\toprule
\textbf{功能} & \textbf{Pinocchio} & \textbf{MuJoCo} \\
\midrule
\endhead
\bottomrule
\endfoot
\bottomrule
\endlastfoot
关节体算法 (ABA) & \texttt{pin.aba(model, data, q, v, tau, [f\_ext])} & \texttt{mujoco.mj\_forward(model, data)} \\
 & \texttt{→ 返回加速度} & \texttt{+ mujoco.mj\_step(model, data)} \\
质量矩阵 & \texttt{pin.crba(model, data, q)} & \texttt{mujoco.mj\_fullM(model, M, data.qM)} \\
 & \texttt{→ data.M} & \\
质量矩阵逆 & \texttt{pin.computeMinverse(model, data, q)} & \texttt{mujoco.mj\_solveLD(model, data, ...)} \\
 & \texttt{→ data.Minv} & \\
\bottomrule
\end{longtable}

\subsection{逆向动力学}

\begin{longtable}{p{5cm}p{6cm}p{6cm}}
\toprule
\textbf{功能} & \textbf{Pinocchio} & \textbf{MuJoCo} \\
\midrule
\endfirsthead
\toprule
\textbf{功能} & \textbf{Pinocchio} & \textbf{MuJoCo} \\
\midrule
\endhead
\bottomrule
\endfoot
\bottomrule
\endlastfoot
递归牛顿-欧拉 (RNEA) & \texttt{pin.rnea(model, data, q, v, a, [f\_ext])} & \texttt{mujoco.mj\_inverse(model, data)} \\
 & \texttt{→ 返回扭矩} & \texttt{→ data.qfrc\_inverse} \\
广义重力 & \texttt{pin.computeGeneralizedGravity(model, data, q)} & \texttt{mujoco.mj\_inverse(model, data)} \\
 & & \texttt{(设置 v=0, a=0)} \\
\bottomrule
\end{longtable}

\section{导数计算 API}

\subsection{解析导数(Pinocchio 独有)}

\begin{longtable}{p{5cm}p{6cm}}
\toprule
\textbf{功能} & \textbf{Pinocchio API} \\
\midrule
\endfirsthead
\toprule
\textbf{功能} & \textbf{Pinocchio API} \\
\midrule
\endhead
\bottomrule
\endfoot
\bottomrule
\endlastfoot
RNEA 导数 & \texttt{pin.computeRNEADerivatives(model, data, q, v, a, [f\_ext])} \\
 & \texttt{→ dtau\_dq, dtau\_dv, dtau\_da} \\
正向运动学导数 & \texttt{pin.computeForwardKinematicsDerivatives(model, data, q, v, a)} \\
速度导数 & \texttt{pin.getJointVelocityDerivatives(model, data, joint\_id, ref\_frame)} \\
加速度导数 & \texttt{pin.getJointAccelerationDerivatives(model, data, joint\_id, ref\_frame)} \\
框架速度导数 & \texttt{pin.getFrameVelocityDerivatives(model, data, frame\_id, ref\_frame)} \\
框架加速度导数 & \texttt{pin.getFrameAccelerationDerivatives(model, data, frame\_id, ref\_frame)} \\
配置积分导数 & \texttt{pin.dIntegrate(model, q, v, [arg\_pos])} \\
配置差分导数 & \texttt{pin.dDifference(model, q1, q2, [arg\_pos])} \\
\bottomrule
\end{longtable}

\textbf{注意}: MuJoCo 不提供解析导数,需要使用有限差分或自动微分库(如 JAX)。

\section{碰撞检测 API}

\begin{longtable}{p{5cm}p{6cm}p{6cm}}
\toprule
\textbf{功能} & \textbf{Pinocchio} & \textbf{MuJoCo} \\
\midrule
\endfirsthead
\toprule
\textbf{功能} & \textbf{Pinocchio} & \textbf{MuJoCo} \\
\midrule
\endhead
\bottomrule
\endfoot
\bottomrule
\endlastfoot
更新几何位置 & \texttt{pin.updateGeometryPlacements(model, data, geometry\_model, geometry\_data, [q])} & \texttt{mujoco.mj\_forward(model, data)} \\
碰撞检测(所有对) & \texttt{pin.computeCollisions(model, data, geometry\_model, geometry\_data, q)} & \texttt{mujoco.mj\_collision(model, data)} \\
 & & \texttt{→ data.ncon, data.contact} \\
碰撞检测(单对) & \texttt{pin.computeCollisions(geometry\_model, geometry\_data, pair\_index)} & \texttt{检查 data.contact 数组} \\
距离计算 & \texttt{pin.computeDistance(geometry\_model, geometry\_data, [pair\_index])} & \texttt{mujoco.mj\_ray(model, data, ...)} \\
\bottomrule
\end{longtable}

\section{质心和能量 API}

\subsection{质心}

\begin{longtable}{p{5cm}p{6cm}p{6cm}}
\toprule
\textbf{功能} & \textbf{Pinocchio} & \textbf{MuJoCo} \\
\midrule
\endfirsthead
\toprule
\textbf{功能} & \textbf{Pinocchio} & \textbf{MuJoCo} \\
\midrule
\endhead
\bottomrule
\endfoot
\bottomrule
\endlastfoot
质心位置 & \texttt{pin.centerOfMass(model, data, q, [v, a])} & \texttt{data.subtree\_com[0]} \\
 & \texttt{→ 返回 COM} & \texttt{(根节点质心)} \\
质心雅可比 & \texttt{pin.jacobianCenterOfMass(model, data, [q])} & \texttt{mujoco.mj\_jacBodyCom(model, data, jacp, jacr, 0)} \\
质心动量 & \texttt{pin.computeCentroidalMomentum(model, data, [q, v])} & \texttt{data.cinert[...]} \\
 & \texttt{→ data.hg} & \texttt{(需要计算)} \\
总质量 & \texttt{pin.computeTotalMass(model, [data])} & \texttt{model.body\_mass.sum()} \\
\bottomrule
\end{longtable}

\subsection{能量}

\begin{longtable}{p{5cm}p{6cm}p{6cm}}
\toprule
\textbf{功能} & \textbf{Pinocchio} & \textbf{MuJoCo} \\
\midrule
\endfirsthead
\toprule
\textbf{功能} & \textbf{Pinocchio} & \textbf{MuJoCo} \\
\midrule
\endhead
\bottomrule
\endfoot
\bottomrule
\endlastfoot
动能 & \texttt{pin.computeKineticEnergy(model, data, [q, v])} & \texttt{data.energy[0]} \\
势能 & \texttt{pin.computePotentialEnergy(model, data, [q, v])} & \texttt{data.energy[1]} \\
机械能 & \texttt{pin.computeMechanicalEnergy(model, data, [q, v])} & \texttt{data.energy[0] + data.energy[1]} \\
\bottomrule
\end{longtable}

\section{接触和约束 API}

\subsection{接触动力学}

\begin{longtable}{p{5cm}p{6cm}p{6cm}}
\toprule
\textbf{功能} & \textbf{Pinocchio} & \textbf{MuJoCo} \\
\midrule
\endfirsthead
\toprule
\textbf{功能} & \textbf{Pinocchio} & \textbf{MuJoCo} \\
\midrule
\endhead
\bottomrule
\endfoot
\bottomrule
\endlastfoot
约束雅可比 & \texttt{pin.getConstraintJacobian(model, data, contact\_model, contact\_data)} & \texttt{从 data.contact 提取} \\
带接触的正向动力学 & \texttt{pin.forwardDynamics(model, data, q, v, tau, constraint\_jacobian, ...)} & \texttt{mujoco.mj\_forward(model, data)} \\
 & & \texttt{+ mujoco.mj\_step(model, data)} \\
冲击动力学 & \texttt{pin.impulseDynamics(model, data, q, v, contact\_models, ...)} & \texttt{mujoco.mj\_step2(model, data)} \\
约束动力学 & \texttt{pin.constraintDynamics(model, data, q, v, tau, contact\_models, ...)} & \texttt{内置在 mj\_step 中} \\
\bottomrule
\end{longtable}

\section{回归器 API(Pinocchio 独有)}

\begin{longtable}{p{5cm}p{6cm}}
\toprule
\textbf{功能} & \textbf{Pinocchio API} \\
\midrule
\endfirsthead
\toprule
\textbf{功能} & \textbf{Pinocchio API} \\
\midrule
\endhead
\bottomrule
\endfoot
\bottomrule
\endlastfoot
静态回归器 & \texttt{pin.computeStaticRegressor(model, data, q)} \\
物体回归器 & \texttt{pin.bodyRegressor(velocity, acceleration)} \\
关节扭矩回归器 & \texttt{pin.computeJointTorqueRegressor(model, data, q, v, a)} \\
运动学回归器 & \texttt{pin.computeJointKinematicRegressor(model, data, joint\_id, ...)} \\
框架回归器 & \texttt{pin.computeFrameKinematicRegressor(model, data, frame\_id, ...)} \\
\bottomrule
\end{longtable}

\textbf{注意}: MuJoCo 不提供回归器功能,主要用于仿真而非参数辨识。

\section{可视化 API}

\subsection{可视化(MuJoCo 独有)}

\begin{longtable}{p{5cm}p{6cm}}
\toprule
\textbf{功能} & \textbf{MuJoCo API} \\
\midrule
\endfirsthead
\toprule
\textbf{功能} & \textbf{MuJoCo API} \\
\midrule
\endhead
\bottomrule
\endfoot
\bottomrule
\endlastfoot
创建查看器 & \texttt{viewer = mujoco.Viewer(model)} \\
渲染 & \texttt{mujoco.mj\_render(model, data)} \\
同步查看器 & \texttt{viewer.sync()} \\
显示 & \texttt{viewer.render()} \\
\bottomrule
\end{longtable}

\textbf{注意}: Pinocchio 不提供内置可视化,需要使用外部工具(如 Meshcat、Gepetto Viewer)。

\subsection{Pinocchio 可视化(外部工具)}

\begin{longtable}{p{5cm}p{6cm}}
\toprule
\textbf{功能} & \textbf{Pinocchio + Meshcat} \\
\midrule
\endfirsthead
\toprule
\textbf{功能} & \textbf{Pinocchio + Meshcat} \\
\midrule
\endhead
\bottomrule
\endfoot
\bottomrule
\endlastfoot
创建可视化器 & \texttt{viz = pin.visualize.MeshcatVisualizer(model, collision\_model, visual\_model)} \\
初始化 & \texttt{viz.initViewer(loadModel=True)} \\
显示 & \texttt{viz.display(q)} \\
\bottomrule
\end{longtable}

\section{代码生成 API(Pinocchio 独有)}

\begin{longtable}{p{5cm}p{6cm}}
\toprule
\textbf{功能} & \textbf{Pinocchio API} \\
\midrule
\endfirsthead
\toprule
\textbf{功能} & \textbf{Pinocchio API} \\
\midrule
\endhead
\bottomrule
\endfoot
\bottomrule
\endlastfoot
代码生成 & 使用 CppADCodeGen 生成 C/C++ 代码 \\
 & 运行时生成针对特定机器人的优化代码 \\
 & 性能提升 3-8 倍 \\
\bottomrule
\end{longtable}

\textbf{注意}: MuJoCo 不提供代码生成功能。

\section{完整示例对比}

\subsection{正向动力学示例}

\textbf{Pinocchio:}
\begin{lstlisting}
import pinocchio as pin
import numpy as np

# 加载模型
model = pin.buildModelFromUrdf("robot.urdf")
data = model.createData()

# 设置状态
q = pin.neutral(model)
v = np.zeros(model.nv)
tau = np.zeros(model.nv)

# 计算正向动力学
a = pin.aba(model, data, q, v, tau)
\end{lstlisting}

\textbf{MuJoCo:}
\begin{lstlisting}
import mujoco
import numpy as np

# 加载模型
model = mujoco.MjModel.from_xml_path("robot.xml")
data = mujoco.MjData(model)

# 设置状态
data.qpos[:] = model.qpos0
data.qvel[:] = np.zeros(model.nv)
data.ctrl[:] = np.zeros(model.nu)

# 计算正向动力学
mujoco.mj_forward(model, data)
mujoco.mj_step(model, data)
# 加速度在 data.qacc
\end{lstlisting}

\subsection{逆向动力学示例}

\textbf{Pinocchio:}
\begin{lstlisting}
# 设置配置、速度、加速度
q = pin.randomConfiguration(model)
v = np.random.rand(model.nv)
a = np.random.rand(model.nv)

# 计算所需扭矩
tau = pin.rnea(model, data, q, v, a)
\end{lstlisting}

\textbf{MuJoCo:}
\begin{lstlisting}
# 设置配置、速度、加速度
data.qpos[:] = q
data.qvel[:] = v
data.qacc[:] = a

# 计算所需扭矩
mujoco.mj_inverse(model, data)
tau = data.qfrc_inverse.copy()
\end{lstlisting}

\section{API 设计哲学对比}

\subsection{共同点}

\begin{itemize}
    \item \textbf{Model-Data 分离}: 两者都采用这种设计模式,提高效率和并行性
    \item \textbf{函数式 API}: 大多数操作都是函数调用,而非面向对象方法
    \item \textbf{状态管理}: 状态信息存储在 data 对象中
\end{itemize}

\subsection{差异点}

\begin{itemize}
    \item \textbf{Pinocchio}: 
        \begin{itemize}
            \item 函数返回计算结果(如 \texttt{tau = pin.rnea(...)})
            \item 更明确的函数命名(如 \texttt{forwardKinematics})
            \item 支持多种参考坐标系(LOCAL, WORLD, LOCAL\_WORLD\_ALIGNED)
        \end{itemize}
    \item \textbf{MuJoCo}:
        \begin{itemize}
            \item 结果存储在 data 中(如 \texttt{data.qfrc\_inverse})
            \item 更简洁的函数名(如 \texttt{mj\_forward})
            \item 内置物理仿真循环
        \end{itemize}
\end{itemize}

\section{总结}

\begin{itemize}
    \item \textbf{Pinocchio} 专注于算法和优化,提供解析导数和代码生成
    \item \textbf{MuJoCo} 专注于仿真和可视化,提供完整的物理引擎
    \item 两者 API 设计相似,但功能定位不同
    \item 可以根据需求选择或结合使用
\end{itemize}

\end{document}

