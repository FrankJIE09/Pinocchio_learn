\documentclass[12pt,a4paper]{article}
\usepackage[UTF8]{ctex}
\usepackage{geometry}
\usepackage{amsmath}
\usepackage{amssymb}
\usepackage{listings}
\usepackage{xcolor}
\usepackage{hyperref}
\usepackage{tikz}

\geometry{margin=2cm}

\lstset{
    language=Python,
    basicstyle=\ttfamily\small,
    keywordstyle=\color{blue}\ttfamily,
    stringstyle=\color{red}\ttfamily,
    commentstyle=\color{green}\ttfamily,
    breaklines=true,
    frame=single,
    numbers=left,
    numberstyle=\tiny\color{gray},
    tabsize=2
}

\title{解析导数(Analytical Derivatives)详解}
\author{机器人动力学中的梯度计算}
\date{}

\begin{document}

\maketitle

\section{什么是解析导数?}

\subsection{基本概念}

\textbf{解析导数(Analytical Derivatives)}是指通过数学公式直接计算函数的导数,而不是通过数值方法(如有限差分)来近似。

\subsection{简单例子}

考虑一个简单的函数:
\begin{equation}
f(x) = x^2 + 3x + 5
\end{equation}

\textbf{解析导数}(通过数学公式):
\begin{equation}
\frac{df}{dx} = 2x + 3
\end{equation}

在 $x = 5$ 处:
\begin{equation}
\frac{df}{dx}\Big|_{x=5} = 2 \times 5 + 3 = 13
\end{equation}

\textbf{数值导数}(通过有限差分近似):
\begin{equation}
\frac{df}{dx}\Big|_{x=5} \approx \frac{f(5+h) - f(5)}{h}
\end{equation}

其中 $h$ 是一个很小的数(如 $h = 0.0001$)。

\section{解析导数 vs 数值导数}

\subsection{对比表}

\begin{table}[h]
\centering
\begin{tabular}{|p{5cm}|p{6cm}|p{6cm}|}
\hline
\textbf{特性} & \textbf{解析导数} & \textbf{数值导数} \\
\hline
计算方法 & 数学公式直接计算 & 有限差分近似 \\
\hline
精度 & 精确(机器精度) & 近似(有截断误差) \\
\hline
计算速度 & 快 & 慢(需要多次函数评估) \\
\hline
数值稳定性 & 稳定 & 可能不稳定($h$ 的选择) \\
\hline
实现复杂度 & 需要推导公式 & 实现简单 \\
\hline
\end{tabular}
\caption{解析导数与数值导数对比}
\end{table}

\subsection{数值导数的局限性}

数值导数使用有限差分方法:
\begin{equation}
\frac{\partial f}{\partial x_i} \approx \frac{f(x + h e_i) - f(x)}{h}
\end{equation}

其中 $e_i$ 是第 $i$ 个单位向量。

\textbf{问题}:
\begin{itemize}
    \item 需要 $n+1$ 次函数评估($n$ 是变量维度)
    \item 精度受 $h$ 选择影响:$h$ 太小→数值误差,$h$ 太大→截断误差
    \item 计算成本高,特别是高维问题
\end{itemize}

\section{机器人动力学中的解析导数}

\subsection{为什么需要导数?}

在机器人学中,导数(梯度)对于以下应用至关重要:

\begin{enumerate}
    \item \textbf{轨迹优化}:需要目标函数关于配置的梯度
    \item \textbf{模型预测控制(MPC)}:需要动力学模型的雅可比矩阵
    \item \textbf{参数辨识}:需要回归器矩阵
    \item \textbf{最优控制}:需要哈密顿函数的梯度
\end{enumerate}

\subsection{具体例子:逆向动力学的导数}

逆向动力学函数:
\begin{equation}
\tau = \text{RNEA}(q, v, a)
\end{equation}

其中:
\begin{itemize}
    \item $q$:关节配置($n$ 维)
    \item $v$:关节速度($n$ 维)
    \item $a$:关节加速度($n$ 维)
    \item $\tau$:关节扭矩($n$ 维)
\end{itemize}

\textbf{解析导数}计算:
\begin{align}
\frac{\partial \tau}{\partial q} &\in \mathbb{R}^{n \times n} \quad \text{(扭矩关于配置的雅可比)} \\
\frac{\partial \tau}{\partial v} &\in \mathbb{R}^{n \times n} \quad \text{(扭矩关于速度的雅可比)} \\
\frac{\partial \tau}{\partial a} &\in \mathbb{R}^{n \times n} \quad \text{(扭矩关于加速度的雅可比)}
\end{align}

\subsection{数值方法 vs 解析方法}

\textbf{数值方法}(有限差分):
\begin{lstlisting}
def numerical_derivative_rnea(model, data, q, v, a):
    n = model.nv
    dtau_dq = np.zeros((n, n))
    
    # 对每个配置变量
    for i in range(n):
        h = 1e-6
        q_plus = q.copy()
        q_plus[i] += h
        
        tau_plus = pin.rnea(model, data, q_plus, v, a)
        tau = pin.rnea(model, data, q, v, a)
        
        dtau_dq[:, i] = (tau_plus - tau) / h
    
    return dtau_dq
\end{lstlisting}

\textbf{问题}:
\begin{itemize}
    \item 需要 $n+1$ 次 RNEA 调用
    \item 对于 30 自由度机器人,需要 31 次调用
    \item 每次调用约 10-50 $\mu$s,总计约 0.3-1.5 ms
\end{itemize}

\textbf{解析方法}(Pinocchio):
\begin{lstlisting}
# 一次调用计算所有导数
dtau_dq, dtau_dv, dtau_da = pin.computeRNEADerivatives(
    model, data, q, v, a
)
\end{lstlisting}

\textbf{优势}:
\begin{itemize}
    \item 只需 1 次函数调用
    \item 计算时间约 20-100 $\mu$s(比数值方法快 10-50 倍)
    \item 精度高(机器精度)
\end{itemize}

\section{Pinocchio 中的解析导数}

\subsection{可用的解析导数}

Pinocchio 提供以下函数的解析导数:

\begin{enumerate}
    \item \textbf{逆向动力学(RNEA)}:
        \begin{itemize}
            \item $\frac{\partial \tau}{\partial q}$, $\frac{\partial \tau}{\partial v}$, $\frac{\partial \tau}{\partial a}$
        \end{itemize}
    
    \item \textbf{正向运动学}:
        \begin{itemize}
            \item $\frac{\partial v}{\partial q}$, $\frac{\partial v}{\partial \dot{q}}$(速度关于配置和速度的导数)
            \item $\frac{\partial a}{\partial q}$, $\frac{\partial a}{\partial \dot{q}}$, $\frac{\partial a}{\partial \ddot{q}}$(加速度的导数)
        \end{itemize}
    
    \item \textbf{框架雅可比}:
        \begin{itemize}
            \item $\frac{\partial J}{\partial q}$(雅可比关于配置的导数)
        \end{itemize}
    
    \item \textbf{配置操作}:
        \begin{itemize}
            \item $\frac{\partial}{\partial q} \text{integrate}(q, v)$(积分的导数)
            \item $\frac{\partial}{\partial q} \text{difference}(q_1, q_2)$(差分的导数)
        \end{itemize}
\end{enumerate}

\subsection{API 示例}

\begin{lstlisting}
import pinocchio as pin
import numpy as np

# 加载模型
model = pin.buildModelFromUrdf("robot.urdf")
data = model.createData()

# 设置状态
q = pin.randomConfiguration(model)
v = np.random.rand(model.nv)
a = np.random.rand(model.nv)

# 计算逆向动力学及其导数
tau = pin.rnea(model, data, q, v, a)

# 解析导数(一次调用)
dtau_dq, dtau_dv, dtau_da = pin.computeRNEADerivatives(
    model, data, q, v, a
)

# dtau_dq: (nv, nv) 矩阵
# dtau_dv: (nv, nv) 矩阵  
# dtau_da: (nv, nv) 矩阵(通常是质量矩阵)
\end{lstlisting}

\section{解析导数的数学原理}

\subsection{链式法则}

解析导数的计算基于链式法则。以 RNEA 为例:

RNEA 算法包含两个阶段:
\begin{enumerate}
    \item \textbf{前向传递}:计算每个连杆的速度和加速度
    \item \textbf{后向传递}:计算力和扭矩
\end{enumerate}

每个阶段都有解析导数,通过链式法则组合:
\begin{equation}
\frac{\partial \tau}{\partial q} = \frac{\partial \tau}{\partial \text{forces}} \cdot \frac{\partial \text{forces}}{\partial \text{accelerations}} \cdot \frac{\partial \text{accelerations}}{\partial q}
\end{equation}

\subsection{递归计算}

Pinocchio 使用递归算法计算导数,类似于动力学算法本身:

\begin{lstlisting}
# 伪代码:RNEA 导数的递归计算
def compute_rnea_derivatives(model, data, q, v, a):
    # 前向传递:计算速度和加速度及其导数
    for i in range(model.njoints):
        # 计算速度导数
        dv_dq[i] = compute_velocity_derivative(...)
        dv_dv[i] = compute_velocity_derivative(...)
        
        # 计算加速度导数
        da_dq[i] = compute_acceleration_derivative(...)
        da_dv[i] = compute_acceleration_derivative(...)
        da_da[i] = compute_acceleration_derivative(...)
    
    # 后向传递:计算力和扭矩及其导数
    for i in range(model.njoints-1, -1, -1):
        # 计算力导数
        df_dq[i] = compute_force_derivative(...)
        # ...
        
        # 计算扭矩导数
        dtau_dq[i] = extract_torque_derivative(...)
        # ...
\end{lstlisting}

\section{性能对比}

\subsection{计算时间对比}

对于 30 自由度的机器人:

\begin{table}[h]
\centering
\begin{tabular}{|l|c|c|}
\hline
\textbf{方法} & \textbf{函数调用次数} & \textbf{计算时间} \\
\hline
数值导数(有限差分) & 31 & $\sim$ 0.5-1.5 ms \\
\hline
解析导数(Pinocchio) & 1 & $\sim$ 0.02-0.1 ms \\
\hline
\textbf{加速比} & \textbf{31x} & \textbf{10-50x} \\
\hline
\end{tabular}
\caption{30 自由度机器人的导数计算性能对比}
\end{table}

\subsection{精度对比}

\begin{table}[h]
\centering
\begin{tabular}{|l|c|}
\hline
\textbf{方法} & \textbf{精度} \\
\hline
数值导数(有限差分) & $O(h)$ 或 $O(h^2)$(取决于方法) \\
\hline
解析导数 & 机器精度($\sim 10^{-15}$) \\
\hline
\end{tabular}
\caption{精度对比}
\end{table}

\section{应用场景}

\subsection{轨迹优化}

在轨迹优化中,需要最小化目标函数:
\begin{equation}
J(q(t)) = \int_0^T L(q(t), \dot{q}(t), \ddot{q}(t)) dt
\end{equation}

使用梯度下降或牛顿法需要梯度:
\begin{equation}
\frac{\partial J}{\partial q} = \int_0^T \frac{\partial L}{\partial q} + \frac{\partial L}{\partial \dot{q}} \frac{\partial \dot{q}}{\partial q} + \frac{\partial L}{\partial \ddot{q}} \frac{\partial \ddot{q}}{\partial q} dt
\end{equation}

解析导数使得梯度计算快速且精确。

\subsection{模型预测控制(MPC)}

MPC 需要预测模型的线性化:
\begin{equation}
\delta \tau = \frac{\partial \tau}{\partial q} \delta q + \frac{\partial \tau}{\partial v} \delta v + \frac{\partial \tau}{\partial a} \delta a
\end{equation}

解析导数提供精确的线性化,提高 MPC 的稳定性和性能。

\subsection{参数辨识}

使用回归器进行参数辨识:
\begin{equation}
\tau = Y(q, v, a) \theta
\end{equation}

其中 $Y$ 是回归器矩阵,$\theta$ 是待辨识参数。解析导数用于计算回归器矩阵的梯度。

\section{为什么 MuJoCo 不提供解析导数?}

\begin{itemize}
    \item \textbf{定位不同}:MuJoCo 专注于仿真,而非优化
    \item \textbf{实现复杂度}:解析导数需要大量数学推导和代码实现
    \item \textbf{使用场景}:强化学习通常不需要解析导数(使用策略梯度等方法)
    \item \textbf{替代方案}:可以使用 JAX 等自动微分库
\end{itemize}

\section{总结}

\begin{itemize}
    \item \textbf{解析导数}是通过数学公式直接计算导数,而非数值近似
    \item \textbf{优势}:速度快、精度高、数值稳定
    \item \textbf{应用}:轨迹优化、MPC、参数辨识等需要梯度的场景
    \item \textbf{Pinocchio 的优势}:提供完整的解析导数支持,这是其核心特性之一
    \item \textbf{性能}:比数值方法快 10-50 倍,精度达到机器精度
\end{itemize}

解析导数是 Pinocchio 相比其他动力学库(如 MuJoCo、RBDL)的独特优势,使其特别适合需要梯度的优化和控制应用。

\end{document}

