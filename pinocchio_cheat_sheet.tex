\documentclass[10pt,a4paper]{article}
\usepackage[UTF8]{ctex}
\usepackage{geometry}
\usepackage{booktabs}
\usepackage{longtable}
\usepackage{array}
\usepackage{listings}
\usepackage{xcolor}
\usepackage{hyperref}

\geometry{margin=1cm}

% 代码样式设置
\lstset{
    language=Python,
    basicstyle=\ttfamily\small,
    keywordstyle=\color{blue}\ttfamily,
    stringstyle=\color{red}\ttfamily,
    commentstyle=\color{green}\ttfamily,
    breaklines=true,
    frame=single,
    numbers=left,
    numberstyle=\tiny\color{gray},
    tabsize=2
}

\title{Pinocchio 速查表}
\author{Pinocchio Library}
\date{}

\begin{document}

\maketitle

\section{快速开始}

\begin{longtable}{p{4cm}p{8cm}}
\toprule
\textbf{功能} & \textbf{代码} \\
\midrule
\endfirsthead
\toprule
\textbf{功能} & \textbf{代码} \\
\midrule
\endhead
\bottomrule
\endfoot
\bottomrule
\endlastfoot
简单安装 & \texttt{conda install -c conda-forge pinocchio} \\
导入库 & \texttt{import pinocchio as pin} \\
文档 & \texttt{from pinocchio.utils import *} \\
 & \texttt{pin.Model?} \\
\bottomrule
\end{longtable}

\section{空间量}

\subsection{变换 (SE3)}

\begin{longtable}{p{4cm}p{8cm}}
\toprule
\textbf{功能} & \textbf{代码} \\
\midrule
\endfirsthead
\toprule
\textbf{功能} & \textbf{代码} \\
\midrule
\endhead
\bottomrule
\endfoot
\bottomrule
\endlastfoot
SE3 变换 & \texttt{aMb = pin.SE3(aRb, apb)} \\
单位变换 & \texttt{M = pin.SE3(1) or pin.SE3.Identity()} \\
随机变换 & \texttt{pin.SE3.Random()} \\
旋转矩阵 & \texttt{M.rotation} \\
平移向量 & \texttt{M.translation} \\
SE3 逆变换 & \texttt{bMa = aMb.inverse()} \\
SE3 作用 & \texttt{aMc = aMb * bMc} \\
作用矩阵 & \texttt{aXb = aMb.action} \\
齐次矩阵 & \texttt{aHb = aMb.homogeneous} \\
对数运算 SE3 → 6D & \texttt{pin.log(M)} \\
指数运算 & \texttt{pin.exp(M)} \\
\bottomrule
\end{longtable}

\subsection{空间速度 (Motion)}

\begin{longtable}{p{4cm}p{8cm}}
\toprule
\textbf{功能} & \textbf{代码} \\
\midrule
\endfirsthead
\toprule
\textbf{功能} & \textbf{代码} \\
\midrule
\endhead
\bottomrule
\endfoot
\bottomrule
\endlastfoot
运动向量 & \texttt{m = pin.Motion(v, w)} \\
线性加速度 & \texttt{m.linear} \\
角加速度 & \texttt{m.angular} \\
SE3 作用 & \texttt{v\_a = aMb * v\_b} \\
\bottomrule
\end{longtable}

\subsection{空间加速度}

\begin{longtable}{p{4cm}p{8cm}}
\toprule
\textbf{功能} & \textbf{代码} \\
\midrule
\endfirsthead
\toprule
\textbf{功能} & \textbf{代码} \\
\midrule
\endhead
\bottomrule
\endfoot
\bottomrule
\endlastfoot
算法中使用 & \texttt{a = ($\dot{\omega}$, $\dot{v}_O$)} \\
获取经典加速度 & \texttt{a' = a + (0, $\omega \times v_O$)} \\
 & \texttt{pin.classicAcceleration(v, a, [aMb])} \\
\bottomrule
\end{longtable}

\subsection{空间力 (Force)}

\begin{longtable}{p{4cm}p{8cm}}
\toprule
\textbf{功能} & \textbf{代码} \\
\midrule
\endfirsthead
\toprule
\textbf{功能} & \textbf{代码} \\
\midrule
\endhead
\bottomrule
\endfoot
\bottomrule
\endlastfoot
力向量 & \texttt{f = pin.Force(l, n)} \\
线性力 & \texttt{f.linear} \\
扭矩 & \texttt{f.angular} \\
SE3 作用 & \texttt{f\_a = aMb * f\_b} \\
\bottomrule
\end{longtable}

\subsection{空间惯性 (Inertia)}

\begin{longtable}{p{4cm}p{8cm}}
\toprule
\textbf{功能} & \textbf{代码} \\
\midrule
\endfirsthead
\toprule
\textbf{功能} & \textbf{代码} \\
\midrule
\endhead
\bottomrule
\endfoot
\bottomrule
\endlastfoot
惯性 & \texttt{Y = pin.Inertia(mass, com, I)} \\
质量 & \texttt{Y.mass} \\
质心位置 & \texttt{Y.lever} \\
转动惯量 & \texttt{Y.inertia} \\
\bottomrule
\end{longtable}

\subsection{几何}

\begin{longtable}{p{4cm}p{8cm}}
\toprule
\textbf{功能} & \textbf{代码} \\
\midrule
\endfirsthead
\toprule
\textbf{功能} & \textbf{代码} \\
\midrule
\endhead
\bottomrule
\endfoot
\bottomrule
\endlastfoot
四元数 & \texttt{quat = pin.Quaternion(R)} \\
轴角 & \texttt{aa = pin.AngleAxis(angle, axis)} \\
\bottomrule
\end{longtable}

\subsection{有用的转换器}

\begin{longtable}{p{4cm}p{8cm}}
\toprule
\textbf{功能} & \textbf{代码} \\
\midrule
\endfirsthead
\toprule
\textbf{功能} & \textbf{代码} \\
\midrule
\endhead
\bottomrule
\endfoot
\bottomrule
\endlastfoot
SE3 → (x,y,z,quat) & \texttt{pin.se3ToXYZQUAT(M)} \\
(x,y,z,quat) → SE3 & \texttt{pin.XYZQUATToSE3(vec)} \\
\bottomrule
\end{longtable}

\section{数据 (Data)}

\begin{longtable}{p{4cm}p{8cm}}
\toprule
\textbf{功能} & \textbf{代码} \\
\midrule
\endfirsthead
\toprule
\textbf{功能} & \textbf{代码} \\
\midrule
\endhead
\bottomrule
\endfoot
\bottomrule
\endlastfoot
与模型相关的数据 & \texttt{data = pin.Data(model)} \\
 & \texttt{data = model.createData()} \\
关节数据 & \texttt{data.joints} \\
关节/框架位置 & \texttt{data.oMi / [data.oMf]} \\
关节速度 & \texttt{data.v} \\
关节加速度 & \texttt{data.a} \\
关节力 & \texttt{data.f} \\
质量矩阵 & \texttt{data.M} \\
非线性效应 & \texttt{data.nle} \\
质心动量 & \texttt{data.hg} \\
质心矩阵 & \texttt{data.Ag} \\
质心惯性 & \texttt{data.Ig} \\
\bottomrule
\end{longtable}

\section{模型 (Model)}

\begin{longtable}{p{4cm}p{8cm}}
\toprule
\textbf{功能} & \textbf{代码} \\
\midrule
\endfirsthead
\toprule
\textbf{功能} & \textbf{代码} \\
\midrule
\endhead
\bottomrule
\endfoot
\bottomrule
\endlastfoot
运动树模型 & \texttt{model = pin.Model()} \\
模型名称 & \texttt{model.name} \\
关节名称 & \texttt{model.names} \\
关节模型 & \texttt{model.joints} \\
关节位置 & \texttt{model.placements} \\
连杆惯性 & \texttt{model.inertias} \\
框架 & \texttt{model.frames} \\
位置变量数量 & \texttt{model.nq} \\
速度变量数量 & \texttt{model.nv} \\
方法 & 使用 \texttt{?} 获取文档和输入参数 \\
添加关节 & \texttt{model.addJoint} \\
附加物体 & \texttt{model.appendBodyToJoint} \\
添加框架 & \texttt{model.addFrame} \\
将子模型附加到父模型 & \texttt{model.appendModel} \\
构建简化模型 & \texttt{model.buildReducedModel} \\
\bottomrule
\end{longtable}

\section{解析器}

\begin{longtable}{p{4cm}p{8cm}}
\toprule
\textbf{功能} & \textbf{代码} \\
\midrule
\endfirsthead
\toprule
\textbf{功能} & \textbf{代码} \\
\midrule
\endhead
\bottomrule
\endfoot
\bottomrule
\endlastfoot
加载 URDF 文件 & \texttt{pin.buildModelFromUrdf(filename, [root\_joint])} \\
加载 SDF 文件 & \texttt{pin.buildModelFromSdf(filename, [root\_joint], root.link.name, parent.guidance)} \\
\bottomrule
\end{longtable}

\section{参考坐标系}

\subsection{坐标系 (CS)}

\begin{longtable}{p{4cm}p{8cm}}
\toprule
\textbf{坐标系} & \textbf{描述} \\
\midrule
\endfirsthead
\toprule
\textbf{坐标系} & \textbf{描述} \\
\midrule
\endhead
\bottomrule
\endfoot
\bottomrule
\endlastfoot
WORLD & 世界坐标系 \\
LOCAL & 关节的局部坐标系 \\
LOCAL\_WORLD\_ALIGNED & 与世界轴对齐的局部坐标系 \\
\bottomrule
\end{longtable}

\section{配置 (Configuration)}

\begin{longtable}{p{4cm}p{8cm}}
\toprule
\textbf{功能} & \textbf{代码} \\
\midrule
\endfirsthead
\toprule
\textbf{功能} & \textbf{代码} \\
\midrule
\endhead
\bottomrule
\endfoot
\bottomrule
\endlastfoot
随机配置 & \texttt{pin.randomConfiguration(model, [lower\_bound, upper\_bound])} \\
中性配置 & \texttt{pin.neutral(model)} \\
归一化配置 & \texttt{pin.normalize(model, q)} \\
配置差 & \texttt{pin.difference(model, q1, q2)} \\
配置距离 & \texttt{pin.distance(model, q1, q2)} \\
配置距离平方 & \texttt{pin.squareDistance(model, q1, q2)} \\
插值配置 & \texttt{pin.interpolate(model, q1, q2, alpha)} \\
积分配置 & \texttt{pin.integrate(model, q, v)} \\
差分的偏导数 & \texttt{pin.dDifference(model, q1, q2, [arg\_pos])} \\
积分的偏导数 & \texttt{pin.dIntegrate(model, q, v, [arg\_pos])} \\
\bottomrule
\end{longtable}

\section{框架 (Frames)}

\begin{longtable}{p{4cm}p{8cm}}
\toprule
\textbf{功能} & \textbf{代码} \\
\midrule
\endfirsthead
\toprule
\textbf{功能} & \textbf{代码} \\
\midrule
\endhead
\bottomrule
\endfoot
\bottomrule
\endlastfoot
所有操作框架的位置 & \texttt{pin.updateFramePlacements(model, data)} \\
当前框架相对于原点的位置 & \texttt{data.oMf} \\
框架速度 & \texttt{pin.getFrameVelocity(model, data, frame\_id, ref\_frame)} \\
框架加速度 & \texttt{pin.getFrameAcceleration(model, data, frame\_id, ref\_frame)} \\
框架经典加速度 & \texttt{pin.getFrameClassicalAcceleration(model, data, frame\_id, ref\_frame)} \\
框架位置 & \texttt{pin.framesForwardKinematics(model, data, q)} \\
框架雅可比 & \texttt{pin.computeFrameJacobian(model, data, q, frame\_id, ref\_frame)} \\
框架雅可比时间变化 & \texttt{pin.frameJacobianTimeVariation(model, data, q, v, frame\_id, ref\_frame)} \\
空间速度的偏导数 & \texttt{pin.getFrameVelocityDerivatives(model, data, frame\_id, ref\_frame)} \\
空间速度的偏导数(关节) & \texttt{pin.getFrameVelocityDerivatives(model, data, joint\_id, placement, ref\_frame)} \\
空间加速度的偏导数 & \texttt{pin.getFrameAccelerationDerivatives(model, data, frame\_id, ref\_frame)} \\
空间加速度的偏导数(关节) & \texttt{pin.getFrameAccelerationDerivatives(model, data, joint\_id, placement, ref\_frame)} \\
\bottomrule
\end{longtable}

\section{碰撞 (Collision)}

\begin{longtable}{p{4cm}p{8cm}}
\toprule
\textbf{功能} & \textbf{代码} \\
\midrule
\endfirsthead
\toprule
\textbf{功能} & \textbf{代码} \\
\midrule
\endhead
\bottomrule
\endfoot
\bottomrule
\endlastfoot
更新碰撞对象位置 & \texttt{pin.updateGeometryPlacements(model, data, geometry\_model, geometry\_data, [q])} \\
所有对的碰撞检测 & \texttt{pin.computeCollisions(model, data, geometry\_model, geometry\_data, q)} \\
一对的碰撞检测 & \texttt{pin.computeCollisions(geometry\_model, geometry\_data, pair\_index)} \\
碰撞距离 & \texttt{pin.computeDistance(geometry\_model, geometry\_data, [pair\_index])} \\
每对的碰撞距离 & \texttt{pin.computeDistances([model, data], geometry\_model, geometry\_data, [q])} \\
几何体积半径 & \texttt{pin.computeBodyRadius(model, geometry\_model, geometry\_data)} \\
BroadPhase & \texttt{pin.computeCollisions(broadphase\_manager, callback)} \\
 & \texttt{pin.computeCollisions(broadphase\_manager, stop\_at\_first\_collision)} \\
+ 前向运动学更新几何位置 & \texttt{pin.computeCollisions(model, data, broadphase\_manager, q, stop\_at\_first\_collision)} \\
\bottomrule
\end{longtable}

\section{质心 (Center of Mass)}

\begin{longtable}{p{4cm}p{8cm}}
\toprule
\textbf{功能} & \textbf{代码} \\
\midrule
\endfirsthead
\toprule
\textbf{功能} & \textbf{代码} \\
\midrule
\endhead
\bottomrule
\endfoot
\bottomrule
\endlastfoot
模型总质量 & \texttt{pin.computeTotalMass(model, [data])} \\
每个子树的质量 & \texttt{pin.computeSubtreeMasses(model, data)} \\
质心 (COM) & \texttt{pin.centerOfMass(model, data, q, [v, a], [compute\_subtree\_com])} \\
质心雅可比 & \texttt{pin.jacobianCenterOfMass(model, data, [q], [compute\_subtree\_com])} \\
\bottomrule
\end{longtable}

\section{能量 (Energy)}

\begin{longtable}{p{4cm}p{8cm}}
\toprule
\textbf{功能} & \textbf{代码} \\
\midrule
\endfirsthead
\toprule
\textbf{功能} & \textbf{代码} \\
\midrule
\endhead
\bottomrule
\endfoot
\bottomrule
\endlastfoot
前向运动学和动能 & \texttt{pin.computeKineticEnergy(model, data, [q, v])} \\
前向运动学和势能 & \texttt{pin.computePotentialEnergy(model, data, [q, v])} \\
前向运动学和机械能 & \texttt{pin.computeMechanicalEnergy(model, data, [q, v])} \\
\bottomrule
\end{longtable}

\section{运动学 (Kinematics)}

\begin{longtable}{p{4cm}p{8cm}}
\toprule
\textbf{功能} & \textbf{代码} \\
\midrule
\endfirsthead
\toprule
\textbf{功能} & \textbf{代码} \\
\midrule
\endhead
\bottomrule
\endfoot
\bottomrule
\endlastfoot
前向运动学 (FK) & \texttt{pin.forwardKinematics(model, data, q, [v, [a]])} \\
FK 导数 & \texttt{pin.computeForwardKinematicsDerivatives(model, data, q, v, a)} \\
[WORLD] & \texttt{pin.getJointVelocityDerivatives(model, data, joint\_id, pin.ReferenceFrame.WORLD)} \\
[LOCAL] & \texttt{pin.getJointAccelerationDerivatives(model, data, joint\_id, pin.ReferenceFrame.LOCAL)} \\
\bottomrule
\end{longtable}

\section{雅可比 (Jacobian)}

\begin{longtable}{p{4cm}p{8cm}}
\toprule
\textbf{功能} & \textbf{代码} \\
\midrule
\endfirsthead
\toprule
\textbf{功能} & \textbf{代码} \\
\midrule
\endhead
\bottomrule
\endfoot
\bottomrule
\endlastfoot
完整模型雅可比 → data.J & \texttt{pin.computeJointJacobians(model, data, [q])} \\
关节雅可比 & \texttt{pin.getJointJacobian(model, data, joint\_id, ref\_frame)} \\
完整模型 dJ/dt & \texttt{pin.computeJointJacobiansTimeVariation(model, data, q, v)} \\
关节 dJ/dt & \texttt{pin.getJointJacobianTimeVariation(model, data, joint\_id, ref\_frame)} \\
\bottomrule
\end{longtable}

\section{正向动力学 (Forward Dynamics)}

\begin{longtable}{p{4cm}p{8cm}}
\toprule
\textbf{功能} & \textbf{代码} \\
\midrule
\endfirsthead
\toprule
\textbf{功能} & \textbf{代码} \\
\midrule
\endhead
\bottomrule
\endfoot
\bottomrule
\endlastfoot
关节体算法 $\ddot{q}$ & \texttt{pin.aba(model, data, q, v, tau, [f\_ext])} \\
关节空间惯性矩阵逆 & \texttt{pin.computeMinverse(model, data, [q])} \\
复合刚体算法 & \texttt{pin.crba(model, data, q)} \\
\bottomrule
\end{longtable}

\section{逆向动力学 (Inverse Dynamics)}

\begin{longtable}{p{4cm}p{8cm}}
\toprule
\textbf{功能} & \textbf{代码} \\
\midrule
\endfirsthead
\toprule
\textbf{功能} & \textbf{代码} \\
\midrule
\endhead
\bottomrule
\endfoot
\bottomrule
\endlastfoot
递归牛顿-欧拉算法 & \texttt{pin.rnea(model, data, q, v, a, [f\_ext])} \\
广义重力 & \texttt{pin.computeGeneralizedGravity(model, data, q)} \\
dtau\_dq, dtau\_dv, dtau\_da & \texttt{pin.computeRNEADerivatives(model, data, q, v, a, [f\_ext])} \\
\bottomrule
\end{longtable}

\section{质心动力学 (Centroidal)}

\begin{longtable}{p{4cm}p{8cm}}
\toprule
\textbf{功能} & \textbf{代码} \\
\midrule
\endfirsthead
\toprule
\textbf{功能} & \textbf{代码} \\
\midrule
\endhead
\bottomrule
\endfoot
\bottomrule
\endlastfoot
质心动量 & \texttt{pin.computeCentroidalMomentum(model, data, [q, v])} \\
质心动量 + 时间导数 & \texttt{pin.computeCentroidalMomentumTimeVariation(model, data, [q, v, a])} \\
\bottomrule
\end{longtable}

\section{通用 (General)}

\begin{longtable}{p{4cm}p{8cm}}
\toprule
\textbf{功能} & \textbf{代码} \\
\midrule
\endfirsthead
\toprule
\textbf{功能} & \textbf{代码} \\
\midrule
\endhead
\bottomrule
\endfoot
\bottomrule
\endlastfoot
所有项(查看文档) & \texttt{pin.computeAllTerms(model, data, q, v)} \\
\bottomrule
\end{longtable}

\section{回归器 (Regressor)}

\subsection{运动学回归器}

\begin{longtable}{p{4cm}p{8cm}}
\toprule
\textbf{功能} & \textbf{代码} \\
\midrule
\endfirsthead
\toprule
\textbf{功能} & \textbf{代码} \\
\midrule
\endhead
\bottomrule
\endfoot
\bottomrule
\endlastfoot
运动学回归器(关节) & \texttt{pin.computeJointKinematicRegressor(model, data, joint\_id, ref\_frame, [placement])} \\
运动学回归器(框架) & \texttt{pin.computeFrameKinematicRegressor(model, data, frame\_id, ref\_frame)} \\
\bottomrule
\end{longtable}

\subsection{回归器}

\begin{longtable}{p{4cm}p{8cm}}
\toprule
\textbf{功能} & \textbf{代码} \\
\midrule
\endfirsthead
\toprule
\textbf{功能} & \textbf{代码} \\
\midrule
\endhead
\bottomrule
\endfoot
\bottomrule
\endlastfoot
静态回归器 & \texttt{pin.computeStaticRegressor(model, data, q)} \\
物体回归器 & \texttt{pin.bodyRegressor(velocity, acceleration)} \\
附加到关节的物体回归器 & \texttt{pin.jointBodyRegressor(model, data, joint\_id)} \\
附加到框架的物体回归器 & \texttt{pin.frameBodyRegressor(model, data, frame\_id)} \\
关节扭矩回归器 & \texttt{pin.computeJointTorqueRegressor(model, data, q, v, a)} \\
\bottomrule
\end{longtable}

\section{接触雅可比 (Contact Jacobian)}

\begin{longtable}{p{4cm}p{8cm}}
\toprule
\textbf{功能} & \textbf{代码} \\
\midrule
\endfirsthead
\toprule
\textbf{功能} & \textbf{代码} \\
\midrule
\endhead
\bottomrule
\endfoot
\bottomrule
\endlastfoot
约束模型的运动学雅可比 & \texttt{pin.getConstraintJacobian(model, data, contact\_model, contact\_data)} \\
约束模型集的运动学雅可比 & \texttt{pin.getConstraintJacobian(model, data, contact\_models, contact\_datas)} \\
\bottomrule
\end{longtable}

\section{接触动力学 (Contact Dynamics)}

\begin{longtable}{p{4cm}p{8cm}}
\toprule
\textbf{功能} & \textbf{代码} \\
\midrule
\endfirsthead
\toprule
\textbf{功能} & \textbf{代码} \\
\midrule
\endhead
\bottomrule
\endfoot
\bottomrule
\endlastfoot
带接触的约束动力学 & \texttt{pin.forwardDynamics(model, data, [q, v], tau, constraint\_jacobian, constraint\_drift, damping)} \\
带接触的冲击动力学 & \texttt{pin.impulseDynamics(model, data, [q], v\_before, constraint\_jacobian, restitution\_coefficient, damping)} \\
约束矩阵的逆 & \texttt{pin.computeKKTContactDynamicMatrixInverse(model, data, q, constraint\_jac, damping)} \\
\bottomrule
\end{longtable}

\section{约束动力学 (Constraint Dynamics)}

\begin{longtable}{p{4cm}p{8cm}}
\toprule
\textbf{功能} & \textbf{代码} \\
\midrule
\endfirsthead
\toprule
\textbf{功能} & \textbf{代码} \\
\midrule
\endhead
\bottomrule
\endfoot
\bottomrule
\endlastfoot
分配内存 & \texttt{pin.initConstraintDynamics(model, data, contact\_models)} \\
带接触约束的正向动力学 & \texttt{pin.constraintDynamics(model, data, q, v, tau, contact\_models, contact\_datas, [prox\_settings])} \\
带运动学约束的正向动力学导数 & \texttt{pin.computeConstraintDynamicsDerivatives(model, data, contact\_models, contact\_datas, prox\_settings)} \\
\bottomrule
\end{longtable}

\section{冲击动力学 (Impulse Dynamics)}

\begin{longtable}{p{4cm}p{8cm}}
\toprule
\textbf{功能} & \textbf{代码} \\
\midrule
\endfirsthead
\toprule
\textbf{功能} & \textbf{代码} \\
\midrule
\endhead
\bottomrule
\endfoot
\bottomrule
\endlastfoot
带接触约束的冲击动力学 & \texttt{pin.impulseDynamics(model, data, q, v, contact\_models, contact\_datas, r\_coeff, mu)} \\
冲击动力学导数 & \texttt{pin.computeImpulseDynamicsDerivatives(model, data, contact\_models, contact\_datas, r\_coeff, prox\_settings)} \\
\bottomrule
\end{longtable}

\section{Cholesky 分解}

\begin{longtable}{p{4cm}p{8cm}}
\toprule
\textbf{功能} & \textbf{代码} \\
\midrule
\endfirsthead
\toprule
\textbf{功能} & \textbf{代码} \\
\midrule
\endhead
\bottomrule
\endfoot
\bottomrule
\endlastfoot
关节空间惯性矩阵的 Cholesky 分解 & \texttt{pin.cholesky.decompose(model, data)} \\
求解 M x = y 中的 x & \texttt{pin.cholesky.solve(model, data, v)} \\
关节空间惯性矩阵的逆 & \texttt{pin.cholesky.computeMinv(model, data)} \\
\bottomrule
\end{longtable}

\section{可视化器 (Viewer)}

\begin{longtable}{p{4cm}p{8cm}}
\toprule
\textbf{功能} & \textbf{代码} \\
\midrule
\endfirsthead
\toprule
\textbf{功能} & \textbf{代码} \\
\midrule
\endhead
\bottomrule
\endfoot
\bottomrule
\endlastfoot
创建可视化器 & \texttt{mv = pin.visualize.MeshcatVisualizer} \\
加载模型 & \texttt{viz = mv(model, collision\_model, visual\_model)} \\
初始化 & \texttt{viz.initViewer(loadModel=True)} \\
显示 & \texttt{viz.display(q)} \\
添加基本形状 & \\
球体 & \texttt{viz.viewer[name].set\_object(meshcat.geometry.Sphere(size), material)} \\
盒子 & \texttt{viz.viewer[name].set\_object(meshcat.geometry.Box([sizex, sizey, sizez]), material)} \\
更改几何体 [name] 的位置 & \texttt{viz.viewer[name].set\_transform(meshcat.transform(xyzquat\_placement))} \\
\bottomrule
\end{longtable}

\end{document}

